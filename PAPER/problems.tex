\section{The considered initial-boundary value problems}
Waves are very important in the field of fluid dynamics,
acoustics, and many other physical problems as they are an effective way to transmit information (sound, fluid, etc.) across several spatio-temporal scales. 
Our focus is mainly on hydrological applications, however the results can be informative for anyone studying wave dynamics. We focus on two models that describe wave dynamics, namely the 1D acoustic wave and shallow water equations. 
 
\subsection{Wave equation}

A wave equation, modeling the propagation of a wave at a fixed speed is defined as:
\begin{equation}
\label{eq:wave}
\frac{\partial^2u}{\partial t^2} = c^2 \frac{\partial^2u}{\partial x^2},\quad \mbox{where } x \in [x_0, x_{\mathrm{end}}], t \in (0, t_{\mathrm{end}}].
\end{equation}
Here, $u(x, t)$ is the displacement in the second space dimension ($y$-direction) and $c$ is the velocity of wave. The equation is equipped the initial and boundary conditions
\begin{equation}
\label{eq:wave_cond}
u_0 = u(x,0) = u_{\mathrm{in}}(x), \quad
\left.\frac{\partial u(x, t)}{\partial t} \right|_{t=0} = u^{\prime}_{\mathrm{in}}(x), \quad 
u(x_0, t) = u(x_{\mathrm{end}}, t). 
\end{equation}

\subsection{Shallow water equation}

Shallow water equations (SWE) are a set of equations that are derived from physical conservation laws for mass and momentum to describe fluid flow problems. They can be derived by depth averaging 
the Navier--Stokes equations. They are used in predicting cyclones, storm surges, flows around structures etc..

For simplicity, the 1D SWE is used in our study of deep learning techniques. 
This equation is derived from the 2D equations of mass and momentum conservation based on assumption
of incompressibility of water, hydrostatic pressure distribution, and a sufficiently small channel slope. 
The conservative form of 2D SWE takes the following form   
\[
\frac{\partial}{\partial t}Q + \frac{\partial }{\partial t} F(Q) + \frac{\partial }{\partial t} G(Q) = 0,
\]
where $u$ and $v$ are the depth-averaged water velocity in $x$ and $y$ directions, $h$ is the water height 
with respect to a bottom profiled domain as a zero line, and 
$Q = (h, hu, hv)^T$, $F(Q) = (hu, hu^2+ \frac{1}{2}gh^2, huv)^T$, $G(Q) =(hu, huv, hu^2+ \frac{1}{2}gh^2)^T$.
This equation has to be equipped with appropriate initial and boundary conditions. 


The 1D SWE, after having differentiated the terms using the product rule, is written as
\begin{equation}\label{eq:swe}
\begin{array}{rcll}
\displaystyle\frac{\partial h }{\partial t} + u \frac{\partial h }{\partial x} + h \frac{\partial u }{\partial x}  &=& 0
& \mbox{with } x \in [x_0, x_{\mathrm{end}}], t \in (0, t_{\mathrm{end}}],\\[1em]
\displaystyle  h \frac{\partial u }{\partial t} + u \frac{\partial h }{\partial t}  + u^2 \frac{\partial h }{\partial x} + 2uh\frac{\partial u }{\partial x} + g \frac{\partial h }{\partial x} &=& 0 & \mbox{with } x \in [x_0, x_{\mathrm{end}}], t \in (0, t_{\mathrm{end}}],\\[1em]
\end{array}
\end{equation}
where $h$ is the water depth and $u$ is depth-averaged velocity. Hence, in contrast to the wave equation, 
the solution of the 1D SWE is vector-valued. The initial  and boundary 
conditions are given by 
\begin{equation}\label{eq:swe_ic_bc}
u_0 =  u(x,0), \quad  h_0 = h(x,0),\quad 
u_{x_0}=u(x_0,t), \quad  u_{x_{\mathrm{end}}} = u(x_{\mathrm{end}}, t) . 
\end{equation}

We consider the parameterized form of the wave equations described above. A n-th order  partial differential equation along with its initial and boundary conditions can be written as: 
\begin{eqnarray}
  \label{eq:pde}
 \mathcal{F}_{PDE}\left(u, \frac{\partial u}{ \partial t}, \frac{\partial u}{ \partial x}, \frac{{\partial}^2 u}{\partial x^2}, \ldots, \vec{\nu}\right) &=&0 \;  \mbox{for } x \in \Omega, t \in (0,T],\\
\label{eq:ic}
  \mathcal{F}_{IC}\left(u, \frac{\partial u}{ \partial t}, \frac{\partial^2 u}{ \partial t^2}, \ldots\right) &=& 0 \;
 \mbox{for } x \in \Omega \cup \bar{\Omega}, t = 0,\\
\label{eq:bc} \mathcal{F}_{BC}\left(u, \frac{\partial u}{ \partial x}, \frac{\partial^2 u}{ \partial x^2}, \ldots   \right) &=&0 \;  \mbox{for } x \in  \bar{\Omega}, t \in (0,T], 
\end{eqnarray}
where $\Omega$ and $\bar{\Omega}$ denote the spatial domain and its boundary,respectively. 
Here, $\mathcal{F}_{PDE}$ describes the differential operators and $\vec{\nu} = (\nu_1, \nu_2, \ldots)$ 
denotes the parameters of the partial differential equation. We seek to find the solution $u(x,t)$ of the initial-boundary value 
problem with initial conditions and boundary conditions denoted by $\mathcal{F}_{IC}$ and $\mathcal{F}_{BC}$ respectively. 
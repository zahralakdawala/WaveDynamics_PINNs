\section{Introduction}

Partial differential equations (PDEs) are used to mathematically formulate the physical problems and hence, provide us with their solution. The physics of various real life applications comping from physics, biology, electrostatics, finance and other disciplines can be expressed in a complex system of PDEs. Our focus is to investigate the use of neural networks for predicting solutions for wave propogation models (acoustic wave equation and shallow water equation). In most real-scale models, it is either imposiible or infeasible to find an analytical solution for PDEs so we rely heavily on numerical schemes to find solutions. Numerical methods, however, can be very costly in terms of time and space and are proven to be quite inefficient for constantly changing environment. Efficient numerical computation has become increasingly important in this field. The solution is usually sought using one of the classical numerical methods such as Finite Difference, Finite Volume, Finite Element, etc.
\par
\noindent
In the past decade, we have seen machine learning methods such as Gaussian Regression Method, Neural Networks and Deep Learning approaches achieve great results specially in the field of image processing and natural language processing. It has become a widely popular field of research to train machine learning algorithms with physical rules \citep{}. It has been shown \citep{} that if one trains a neural networks with appropriate physics, the engine can be used to predict solutions for complex systems of partial differential equations. A physics informed neural network (PINN) learns the solution to a PDE using the information regarding initial and boundary conditions of the equation in addition to the solution itself. The theory of neural networks state that it can learn any function but as the functions get complicated the number of layers needed to learn that function approaches infinity, which is not something we can apply in practice. PINNs learns the physics and the associated with much less layers than what is required for a conventional network. In addition, it is shown that PINN can also be used to predict solutions of PDEs with slight change in their initial conditions. The issue of efficiency, convergence and accuracy of solutions obtained from using PINNs is explored deeply \citep{}. Regarding this aspect, only a few studies illustrate how the changes in the neural network configurations affect the solution \citep{}, \citep{}. Even fewer report on how the optimization functional can be modified for various physics to get better convergence and accuracy. 
\par
\noindent
This work focuses towards employing neural networks and deep learning frameworks to as an alternative method for finding  solutions of such partial differential equations. We consider the 1D wave equation and 1D Shallow Water Equation (SWE) along with its associated initial and boundary conditions as physical rules to train a neural network. We employ three ways of training the neural networks i) using data from numerical solutions; ii) using only the model/physics as given by the physical equation and its associated initial and boundary conditions; iii) using a hybrid approach that combines both the data and physical model to train the network.  
\par
\noindent
This paper is organised as follows: we first provide a short introduction to the 1D wave and shallow water models considered as the physical models for training of PINNs. This is followed by a description of a vanilla feed forward neural networks. We constructa multi-objective loss function for optimizing the hyperparameters of the network. We further discuss a worked out a worked out example for how the loss function is modified to incorporate the physical model at hand. Section \ref{sec:results} shows the solution and evolution plots for data and physics trained neural networks and compares them against  reference(numerical) solutions. We draw out a detailed discussion of the results in Section \ref{sec:conclusion} summarizing our work and findings.


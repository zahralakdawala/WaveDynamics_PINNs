\section{Conclusion}
\label{sec:conclusion}
First, the convergence of the 
optimization process was studied, where we explored which terms of the multi-objective 
loss functional were easily or hard to minimize. Then, of course, the quality of the solution 
predicted by the networks was studied. On the one hand, we considered exactly the interval were the 
data came from. And on the other hand, we exploreed what happens if a somewhat larger time 
interval is considered, i.e., the predictions of the network were applied to an unseen situation. 

Cant colnclude that PINNS works best for SWE. changing layer and node configuration hardly makes a difference in training the networks. In the case of smooth solutions of wave models, PINNs predicts velocity(density) and/or height rather accurately. However for more complex/coupled systems, data-driven networks seem to work better. However data-fed networks need exhaustive datasets (solutions of model equations). In the absence of exhaustive data for learning of complex solutions, a hybrid training is recommended. 


